\documentclass{beamer}
\usepackage{ctex, hyperref}
\usepackage[T1]{fontenc}

% other packages
\usepackage{latexsym,amsmath,xcolor,multicol,booktabs,calligra}
\usepackage{graphicx,pstricks,listings,stackengine}

\author{张三}
\title{南信大 Beamer 模板}
\subtitle{A NUIST Beamer Template}
\institute{大气物理学院}
\date{2022年1月2日}
\usepackage{NUIST}
\hypersetup{pdfpagemode=FullScreen}
%默认打开即全屏,不需要可以注释掉,仅支持部分阅读器

% defs
\def\cmd#1{\texttt{\color{red}\footnotesize $\backslash$#1}}
\def\env#1{\texttt{\color{blue}\footnotesize #1}}
\definecolor{deepblue}{rgb}{0,0,0.5}
\definecolor{deepred}{rgb}{0.6,0,0}
\definecolor{deepgreen}{rgb}{0,0.5,0}
\definecolor{halfgray}{gray}{0.55}

\lstset{
    basicstyle=\ttfamily\small,
    keywordstyle=\bfseries\color{deepblue},
    emphstyle=\ttfamily\color{deepred},    % Custom highlighting style
    stringstyle=\color{deepgreen},
    numbers=left,
    numberstyle=\small\color{halfgray},
    rulesepcolor=\color{red!20!green!20!blue!20},
    frame=shadowbox,
}


\begin{document}

\kaishu
\begin{frame}
    \titlepage
    \begin{figure}[htpb]
        \begin{center}
            \includegraphics[width=0.2\linewidth]{pic/nuist_logo.eps}
        \end{center}
    \end{figure}
\end{frame}

\begin{frame}{目录页}
    \tableofcontents[sectionstyle=show,subsectionstyle=show/shaded/hide,subsubsectionstyle=show/shaded/hide]
\end{frame}


\section{基本格式设置}

\begin{frame}{逐条放映}
    \begin{itemize}[<+-| alert@+>] % 当然,除了alert,手动在里面插 \pause 也行
        \item 这是一个南信大 \LaTeX{} Beamer主题
        \item 中文支持请选择 Xe\LaTeX{} 编译选项
        \item 改编自BUAA的模板: \url{https://www.overleaf.com/latex/templates/thu-beamer-theme/vwnqmzndvwyb}
        \item 其实就改了个颜色,主题色改为南信大VIS标准中的信大蓝\cite{vis}
    \end{itemize}
\end{frame}

\subsection{这是一个subsection}

\begin{frame}
    \begin{itemize}
        \item 列表实例
        \item 1
        \item 2
    \end{itemize}
\end{frame}


\section{排版示例}

\subsection{表格的排版}

\begin{frame}{Table}
    \begin{itemize}
        \item 如表\ref{tb:one}
    \end{itemize}
    \begin{table}[h]
        \centering
        \begin{tabular}{cc}
        \toprule%第一道横线
        表头1&表头2 \\
        \midrule%第二道横线 
        内容1&内容2 \\
        内容3&内容4 \\
        \bottomrule%第三道横线
        \end{tabular}
        \caption{三线表示例}
        \label{tb:one}
    \end{table}
\end{frame}

\subsection{公式的排版}
\begin{frame}{公式\footnote{推荐一个网站:\url{https://www.latexlive.com/home}}}
    \begin{exampleblock}{无编号公式} % 加 * 
        \begin{equation*}
            P_{r}=\frac{P_{t} \lambda^{2}}{64 \pi^{3} R^{2}}\left[\iint G^{2}(\theta, \varphi) \mathrm{d} \theta \mathrm{d} \varphi\right] \cdot \frac{L}{2} \sum_{i=1}^{N} \sigma_{i}
        \end{equation*}
    \end{exampleblock}
    \begin{exampleblock}{编号公式}
        \begin{align}
            c_{v} \frac{d T}{d t}+p \frac{d \alpha}{d t}=Q
        \end{align}
    \end{exampleblock}
\end{frame}

\begin{frame}[fragile]{\LaTeX{} 常用命令}
    \begin{exampleblock}{命令}
        \centering
        \footnotesize
        \begin{tabular}{llll}
            \cmd{chapter} & \cmd{section} & \cmd{subsection} & \cmd{paragraph} \\
            章 & 节 & 小节 & 带题头段落 \\\hline
            \cmd{centering} & \cmd{emph} & \cmd{verb} & \cmd{url} \\
            居中对齐 & 强调 & 原样输出 & 超链接 \\\hline
            \cmd{footnote} & \cmd{item} & \cmd{caption} & \cmd{includegraphics} \\
            脚注 & 列表条目 & 标题 & 插入图片 \\\hline
            \cmd{label} & \cmd{cite} & \cmd{ref} \\
            标号 & 引用参考文献 & 引用图表公式等\\\hline
        \end{tabular}
    \end{exampleblock}
    \begin{exampleblock}{环境}
        \centering
        \footnotesize
        \begin{tabular}{lll}
            \env{table} & \env{figure} & \env{equation}\\
            表格 & 图片 & 公式 \\\hline
            \env{itemize} & \env{enumerate} & \env{description}\\
            无编号列表 & 编号列表 & 描述 \\\hline
        \end{tabular}
    \end{exampleblock}
\end{frame}

\begin{frame}{作图}
    \begin{itemize}
        \item 矢量图 eps, ps, pdf
        \begin{itemize}
            \item METAPOST, pstricks, pgf $\ldots$
            \item Xfig, Dia, Visio, Inkscape $\ldots$
            \item Matlab / Excel 等保存为 pdf
        \end{itemize}
        \item 标量图 png, jpg, tiff $\ldots$
        \begin{itemize}
            \item 提高清晰度,避免发虚
            \item 应尽量避免使用
        \end{itemize}
    \end{itemize}
    \begin{figure}[htpb]
        \centering
        \includegraphics[width=0.2\linewidth]{pic/nuist_logo.eps}
        \caption{这个校徽就是矢量图}
    \end{figure}
\end{frame}

\section{注意事项}
\begin{frame}{注意事项}
    \begin{itemize}
        \item 适合使用 \LaTeX{} Beamer 的情形:学术报告、汇报会等
        \item 不能使用各种动画效果
        \item 无法进行排练计时
        \item 如果你有以上需求,请使用MS Office
    \end{itemize}
\end{frame}

\section{参考文献}

\begin{frame}[allowframebreaks]
    \bibliography{ref}
    \bibliographystyle{plain}
\end{frame}

\begin{frame}
    \begin{center}
        {\Huge\calligra The End}
    \end{center}
\end{frame}

\end{document}